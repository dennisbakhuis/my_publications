\documentclass[10pt]{article}
\oddsidemargin -0.05in \topmargin -0.25in
\textwidth 6.5in \textheight 8.5in

\usepackage[english]{babel}
\usepackage{graphicx}
\usepackage[amssymb]{SIunits}% SI units package
\usepackage{color}
\usepackage{amsmath}
\usepackage{subcaption}
\renewcommand{\Re}{\mathrm{Re}}
\newcommand{\strong}[1]{\textbf{#1}}
\newcommand{\red}[1]{\textcolor{red}{#1}}
%\newcommand{\makered}[1]{\textbf{#1}}
\newcommand{\question}[1]{\begin{quote} \emph{#1}  \end{quote} }

\newcommand{\comm}[1]{}
\bibliographystyle{jasanum}


\begin{document}

\noindent Date: \today \\
Manuscript: IJMF\_2017\_854\\
Title: Air cavities at the inner cylinder of turbulent Taylor-Couette flow \\
Authors: Ruben A. Verschoof, Dennis Bakhuis, Pim A. Bullee, Sander G. Huisman, Chao Sun, Detlef Lohse

\vspace*{1.25cm}
\section*{Referee 3}
We thank the referee for her/his comments and for the notion that our work is interesting for the community. We react to all comments below.

\question{The paper deals with the influence of air cavities on turbulent Taylor-Couette flow, more specifically on the drag. The air cavities are developping by placing cavitators at the inner cylinder wall of the T3C device. Based on visualizations of the system, the authors extract informations on the size of the cavities and on the air cavity coverage. They show that air cavities can lead to drag reduction, with a larger drag reduction than what have been previously observed by injecting bubbles in the T$^3$C device. This drag reduction is estimated by comparing the torques with and withour air, but always with cavitators. The authors also show that placing such cavitators, without air, induces by itself a drag increase which is larger than what can be later reduced with air cavities. No net drag reduction is therefore observed, on contrary. The author end with a discussion on the possibility to compare the results of the current paper with other configurations such as channel flows. \\
I think this paper is interesting for the community, below are my comments:}

\noindent \strong{1}

\question{The structure of the cavities is not clear to me: I find them difficult to distinguish on figure 4, and I am not sure about their shape. Could the authors also sketch the shape of the cavities on the visualizations, or provide more details in the text?  }

\noindent \strong{Response:} 

\noindent We thank the referee for pointing this out. In the TC setup, illumination is difficult due to the curved cylinders. Light is reflected strongly at the centre of the image, the light intensity is weaker towards the left- and right edges. Therefore, a homogeneous light intensity is very difficult. \\
In the revised manuscript, we improved this section by: i) providing an annotated figure, as suggested. ii) by improving the accompanying text and explanation. And iii) by improving the figures' readability by means of image processing. \\
	 
\noindent \strong{2}

\question{The authors have tested the influence on the number of cavitators and observed similar drag reduction. But how does the number of cavitator affect the length of the cavities and the percentage of coverage? }

\noindent \strong{Response:} 

\noindent  As is shown in fig.\ 6, in almost all cases, the streamwise length is smaller than half of distance between 2 cavities. Therefore, having 2, 3, or 6 cavitators would not change the streamwise cavity length, except for a small part close to the top of the setup at $\alpha=4\%$ when 6 cavitators would be applied. For the case that a cavity extends over multiple cavitators, the submerged cavitators do not significantly affect  the flow or the cavity, as was convincingly shown in ref.\ \cite{Zverkhovskyi2014}. 
We now mention this in our revised manuscript.\\
\\
\noindent \strong{3}

\question{For ventilated cavities, air is injected ``at various heigh''. Where are those injection zones located? Besides, the authors have tested the extent to which the injection of air would influence the drag and no significant modification is observed. But does it influence the characteristics of the cavities?}

\noindent \strong{Response:} 

\noindent We injected air from 4 different heights below the cavitator surface, i.e. at $z/L = \{0.09,~0.33,~0.57,~0.81\}$. Inside the cavitators, we made a small cavity in which the air can spread axially. The influence of air injection to the cavity characteristics was studied in  ref.\ \cite{Zverkhovskyi2014}. Here, it was shown that a surplus in available air (as is the case in our air injection cases) does not lead to longer air cavities or an increased drag reduction. The only direct consequence is an increased amount of discharged air. \\
We now add this notion to the analysis accompanying figure 12. Furthermore, we provide some additional details to the air injection method. 
\\

\noindent \strong{4}

\question{The authors compare the results to the drag reduction obtained with bubble injection, at the same global air volume fraction. In both cases, air is not homogeneously distributed within the gap with a higher gas volume fraction close to the inner cylinder. It would be interesting to compare the results for similar range of local air concentration/coverage at the inner cylinder. }

\noindent \strong{Response:} 

\noindent We agree with the referee on this point: it would be highly interesting, and would shed light on how exactly air cavities and bubbles reduce the drag. Measuring the local void fraction is done in ref. \cite{gil13} by means of sticking a phase-sensitive probe in the flow. We here did not measure the local void fraction. Furthermore,  the local void fraction is an output rather than an input parameter, and its value depends on i) global void fraction, ii) radial position, iii) axial position, iv) Reynolds numbers, and v) Weber numbers. Given this huge parameter space, it is nearly impossible to compare these cases. \\

\noindent \strong{5}

\question{Most of the results presented here concern configurations with stationary outer cylinder, but the visualizations in section 2 indicate that the cases of co and counter-rotation of the outer cylinder have been investigated as well. Is there any modification of the drag to be observed for a rotating outer cylinder? }

\noindent \strong{Response:} 

\noindent The torque measurements for the case of counter-rotating cylinders were already shown in figure 9 of the original manuscript. It was found that counter-rotation results in weaker DR than what was observed for the case of pure inner cylinder rotation. This is connected to the fact that CR cylinders induce Taylor rolls, which enhance mixing between the water and air. Therefore, it is more difficult for air cavities to develop.\\

\noindent \strong{6}

\question{Did the authors try using the cavitators with bubble injection (so filling initially the system entirely with water and then injecting gas)? As the cavitators induce a increase of the drag, it would be interesting to compare how drag can be reduced with bubbles, but starting now from the same reference flow. }

\noindent \strong{Response:} 

\noindent  The procedure we used here focussed on having a constant void fraction $\alpha$ while changing the Reynolds number instead of vice versa. The procedure as suggested by the referee was followed by e.g.\ ref.\ \cite{gil13}. Doing so results in  the disadvantage of being able to study only a limited number of Reynolds numbers, rather than a continuous measurement as we now did. Therefore, we chose to only measure with constant void fractions. \\

\noindent \strong{7}

\question{Minor comments:\\
page 4, last paragraph: some paper references are not strictly speaking correct since not all of those papers deal with drag reduction.\\
Fig.10: caption says dimensionless torque for counter-rotating cylinder }

\noindent \strong{Response:} 

\noindent Page 4: We rewrote this sentence to avoid misunderstandings. \\ Figure 10: We changed this small mistake, and thank the referee for pointing it out.\\




	\bibliography{literatur,2ph_literatur}


\end{document}