\chapter{Summary}
\chaptermark{Summary}
\markboth{\MakeUppercase{Summary}}{}
In many geophysical situations and in all industrial applications,
turbulent flows are wall-bounded. Many of these flows are multi-phase,
\textit{i.e.} flows consisting of one or multiple inclusions.
The current understanding of these flows is still limited and this makes it
important to study them. 
In this thesis we study these wall-bounded multi-phase flows in two
canonical systems: Taylor-Couette flow and Rayleigh-B\'enard convection.\\
% chap 1
\indent First, we used rigid spherical neutrally-buoyant particles 
to investigate if we have reduced skin friction similar to bubbly drag
reduction.
The global torque measurements showed that these particles
barely alter the drag, even at very large particle volume fractions.
One would expect that adding particles increases the apparent viscosity and
therefore, expect an increase in drag.
This is however not found and we hypothesize that the drag reducing effect is
competing with the drag increase from the increased apparent viscosity.\\
% chap 2
\indent In real life, actual bubbles can have any shape, and this might be a key
element in the drag reducing effect.
To test this hypothesis we introduced cylindrical neutrally-buoyant particles
into the system. 
While the drag response of the system was very similar to the spherical
particles, we found that these rods show a preferential alignment with respect
to the inner cylinder wall.
This is very surprising as it was often hypothesized that a systematic
alignment in such highly turbulent flows is not possible.
We model the orientation of the fibers using the Jeffery equations which give
a fair estimate of the shape of the alignment probability density
distributions.\\
% chap 3
\indent Using an immiscible fluid we are able to create deformable inclusions.
We selected silicon oil with a viscosity similar to that of water and varied
the oil volume fraction between 0 and \SI{100}{\percent}.
We found two regimes: first we have oil droplets in water, that gradually
increase the apparent viscosity until the inversion point.
The second regime starts after the inversion where we have water droplets in
oil.
In this regime we find drag reduction and we think that this is due to much
larger water droplets. 
Using an in-situ microscope we were able to confirm this hypothesis as the
water droplets are more than $14\times$ larger than their oil droplet
counterparts.\\
% chap 4
\indent To investigate the effect of non-perfect boundaries, we used another
canonical system: Rayleigh B\'enard Convection (RBC).
Using direct numerical
simulations, the boundaries were divided into equal stripes of conducting
and insulating regions.
While keeping the area identical and varying the amount
of divisions we were able to get almost the same heat transfer as a fully
conducting system. This means that small temperature imperfections are not
visible by the bulk flow.\\
% chap 5
\indent In the same spirit, we applied sand grain roughness onto the inner cylinder of
the Taylor-Couette apparatus, thereby also creating inhomogeneous boundary
conditions. By varying the periodicity of the rough/smooth patches, we were
able to control the secondary flows.
When the width of the roughness patches are similar to the gap size, we find
an optimum in angular momentum transport.
For the largest patches, we can reduce the secondary flows to only two
individual rolls.
This shows that different configurations of roughness can alter the flow
structure tremendously.

