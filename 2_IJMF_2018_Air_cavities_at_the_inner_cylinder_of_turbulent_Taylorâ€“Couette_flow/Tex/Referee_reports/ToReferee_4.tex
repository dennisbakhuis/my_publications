\documentclass[10pt]{article}
\oddsidemargin -0.05in \topmargin -0.25in
\textwidth 6.5in \textheight 8.5in

\usepackage[english]{babel}
\usepackage{graphicx}
\usepackage[amssymb]{SIunits}% SI units package
\usepackage{color}
\usepackage{amsmath}
\usepackage{subcaption}
\renewcommand{\Re}{\mathrm{Re}}
\newcommand{\strong}[1]{\textbf{#1}}
\newcommand{\red}[1]{\textcolor{red}{#1}}
%\newcommand{\makered}[1]{\textbf{#1}}
\newcommand{\question}[1]{\begin{quote} \emph{#1}  \end{quote} }

\newcommand{\comm}[1]{}
\bibliographystyle{jasanum}


\begin{document}

\noindent Date: \today \\
Manuscript: IJMF\_2017\_854\\
Title: Air cavities at the inner cylinder of turbulent Taylor-Couette flow \\
Authors: Ruben A. Verschoof, Dennis Bakhuis, Pim A. Bullee, Sander G. Huisman, Chao Sun, Detlef Lohse

\vspace*{1.25cm}
\section*{Referee 4}
We thank the referee for her/his comments on the quality, novelty and relevance of our work. We thank the referee for his/her comments, which led to significant improvements of our work. See below for our replies to his/her comments.

\question{In this manuscript, the authors perform an experimental study that investigates the introduction of geometric air-cavitators (which facilitates the trapping of air cavities), for the purpose of reducing skin friction drag, in a Taylor-Couette setup. The authors quantitatively measure the air-fraction coverage, air-bubble geometry and net torque in both the co and counter-rotating regime. \vspace{\baselineskip}\\
In my reading of the manuscript, the main interesting results are: \vspace{\baselineskip}\\ 
1. Although there is a net drag increase from pressure/form drag due to the geometry of the cavitators, the air-cavity itself is shown to reduce the ``gross" drag (i.e., associated with skin-friction). The main parameter that controls drag reduction (DR) is ``$\alpha$", the initial volume fraction of air in the TC geometry.\\
2. Interestingly, the air-coverage fraction saturates at large inner Reynolds number, leading to a fixed DR (in contrast to "bubbly" TC flow, where the DR increases with Re).\\
3. In the counter-rotating regime, the air cavity is destabilized due to counter flow, and air bubbles are instead trapped in the taylor vortices that lead to decreased DR.\vspace{\baselineskip}\\
The article is largely well-written and technically sound, although the potential impact suffers from the exploratory nature of the study, and the lack of a net drag reduction relative to the smooth rotor.\vspace{\baselineskip}\\
However, I agree with the authors that studying the interplay between the increase in pressure/form drag and decrease in skin friction drag due to the presence of air-cavitators is an important and relevant topic. In this context, I believe the authors provide valuable new experimental data. \vspace{\baselineskip}\\
I recommend publication after the following minor comments are addressed:}

\noindent \strong{1}

\question{Fig 4 is very unclear, especially in visualizing the location of the air-cavities. Parts of the image are overexposed, and it is difficult to identify the location of the contact line/closure region. It is hard to relate Fig 4a and Fig 4b to the sketch presented in Fig 3. Perhaps the authors can either retake an image with enhanced quality (with better lighting/exposure). Alternatively, if this is difficult due to the nature of the TC setup, I recommend the authors annotate Fig 4 to clearly indicate the locations of the air cavity, air patches and contact line using arrows/lines (as done in Fig 5b). }

\noindent \strong{Response:} 

\noindent We thank the referee for pointing this out. In the TC setup, illumination is difficult due to the curved cylinders. Light is reflected strongly at the centre of the image, the light intensity is weaker towards the left- and right edges. Therefore, a homogeneous light intensity is very difficult. \\
In the revised manuscript, we improved this section by: i) providing an annotated figure, as suggested. ii) by improving the accompanying text and explanation. And iii) by improving the figures' readability by means of image processing. \\
	 
\noindent \strong{2}

\question{Some of the ordering of the presentation can be improved. For example, I think Fig 1 should be referenced in the text while first discussing the concept of the cavitator in Page 3. Similarly, although $\alpha$ is the key parameter in drag reduction in the TC experiment, it is not defined anywhere except in Fig 1B.  }

\noindent \strong{Response:} 

\noindent We thank the referee for pointing this out. We revised the manuscript accordingly.\\

\noindent \strong{3}

\question{The authors state that the DR saturates with increasing Re$_i$ at $\approx 10^6$. However Fig 8b appears to shows that the drag reduction shows a decrease for both $\alpha=2\%$ and $4\%$ at large Re$_i$. Can the authors comment on the uncertainties/fluctuations in measured torque associated with these measurements? Over what time is each measured torque value averaged, and is it possible to include the associated error bars in the DR data? }

\noindent \strong{Response:} 

\noindent In all measurements done with our Taylor-Couette setup, the torque is the primary response parameter. Therefore, we did all efforts to keep the error as small as possible, revising our setup multiple times over the years. Consequently, the error is very small compared to the measurements values. The maximum error in our measurements equals $\tau_{err} = 0.5$ Nm, which equals $G_{err} = 8.5\times 10^7$. \\
We quasi-statically increase the Reynolds number, thus providing a continuous measurement rather than individual torque values. \\
In the revised manuscript, we now added error bars to all figures providing torque, or DR data. We added more details on the torque measuring method in the manuscript.




	%\bibliography{literatur,2ph_literatur}


\end{document}