% arara: pdflatex
\documentclass[tikz,border=0pt]{standalone}
\usepackage{pgfplots}
\usepackage{amsmath}
\usepackage{siunitx}
\usepackage{xcolor}

\usepgfplotslibrary{fillbetween}

% original colors
\definecolor{color1}{RGB}{166,118,29}
\definecolor{color2}{RGB}{217,95,2}
\definecolor{color3}{RGB}{231,41,138}
\definecolor{color4}{RGB}{117,112,179}
\definecolor{color5}{RGB}{27,158,119}
\definecolor{color6}{RGB}{102,166,30}
\definecolor{color7}{RGB}{230,171,2}

% darkend first colors
\definecolor{color1a}{RGB}{116,68,0}
\definecolor{color3a}{RGB}{181,0,88}
\definecolor{color4a}{RGB}{67,62,119}
\definecolor{color6a}{RGB}{52,116,0}

% lightend second colors
\definecolor{color1b}{RGB}{216,168,79}
\definecolor{color3b}{RGB}{255,91,188}
\definecolor{color4b}{RGB}{167,162,229}
\definecolor{color6b}{RGB}{152,216,80}

\definecolor{seq1}{RGB}{186,228,188}
\definecolor{seq2}{RGB}{123,204,196}
\definecolor{seq3}{RGB}{67,162,202}
\definecolor{seq4}{RGB}{8,104,172}

\definecolor{seqa}{RGB}{254,196,79}
\definecolor{seqb}{RGB}{217,95,14}

\definecolor{darkorange}{rgb}{1.0, 0.55, 0.0}

\begin{document}
\begin{tikzpicture}
  [
    rec0/.style={shade,
                rectangle,
                minimum width=6mm,
                minimum height=2.00mm,
                inner sep=0pt,
                outer sep=0pt,
                top color=gray!60!white,
                bottom color=gray!30!white,
                draw=black, 
                very thin},
    rec45/.style={shade,
                rectangle,
                rotate=45,
                minimum width=6mm,
                minimum height=2mm,
                inner sep=0pt,
                outer sep=0pt,
                top color=gray!60!white,
                bottom color=gray!30!white,
                draw=black, 
                very thin},
    rec45r/.style={shade,
                rectangle,
                rotate=-45,
                minimum width=6mm,
                minimum height=2mm,
                inner sep=0pt,
                outer sep=0pt,
                top color=gray!60!white,
                bottom color=gray!30!white,
                draw=black, 
                very thin},
    rec90/.style={shade,
                rectangle,
                rotate=90,
                minimum width=6mm,
                minimum height=2mm,
                inner sep=0pt,
                outer sep=0pt,
                top color=gray!60!white,
                bottom color=gray!30!white,
                draw=black, 
                very thin}
  ]
% width is 8.64cm
\newcommand{\figwidth}{8.64cm}
\newcommand{\figheight}{5.8cm} % Golden ratio

\newcommand{\goldenratio}{1.618033988749895} 
\newcommand{\axiswidth}{7.4cm} 
\newcommand{\axisheight}{\axiswidth / \goldenratio * 0.85} 

\newcommand{\startx}{-0.20cm} 
\newcommand{\starty}{-0.20cm} 


% \draw[use as bounding box, white] (0,0) rectangle (\figwidth,\figheight);
% \draw[red, thin] (0,0) rectangle (\figwidth,\figheight);
  
  \begin{axis}[
    xshift=\startx,
    yshift=\starty,
    clip mode=individual,
    xmin=-90, xmax=90,
    ymin=0.0,
    ymax=0.51,
    enlargelimits=false,
    axis on top=false,
    xlabel=$\theta_p$,
    ylabel=$\text{PDF}(\theta_p)$,
    legend style={draw=none, 
                  font=\normalsize,
                  fill=none,
                  %row sep=-0.9mm,
    },
    legend columns=4, 
    legend style={row sep=2.75pt},    
    legend plot pos=left,
    %legend pos= south west,
    legend style={at={(axis cs: -86, 0.01)}, anchor=south west},
    legend cell align=left,
    legend style={
                % the /tikz/ prefix is necessary here...
                % otherwise, it might end-up with `/pgfplots/column 2`
                % which is not what we want. compare pgfmanual.pdf
        /tikz/column 2/.style={
            column sep=0.48cm,
        },
    },
    scale only axis,
    width = \axiswidth,
    height = \axisheight,
    ylabel near ticks,
    xtick={-90, -45, 0, 45, 90},
    % xticklabels={\ang{-90}, \ang{-45}, \ang{0}, \ang{45}, \ang{90}},
    minor ytick={0.1, 0.3, 0.5},
    minor xtick={-75, -60, -30, -15, 15, 30, 60, 75},
    xticklabels={$-\frac{\pi}{2}$,
                 $-\frac{\pi}{4}$,
                 0,
                 $\frac{\pi}{4}$,
                 $\frac{\pi}{2}$}, 
    xtick style={font=\small},
    ytick style={font=\small},
    xlabel style={yshift=1.5mm},
    ylabel style={yshift=-1.5mm},
  ]


  \addplot+[
    color=seq1,
    mark=none,
    solid,
    very thick,
    line join=round,
   ] table [
    x = bins,
    y = hist,
    col sep=comma,
] {Jeffery2_1.csv};
\addlegendentryexpanded{1.0 $\tau_\ell$}


  \addplot+[
    color=seq2,
    mark=none,
    solid,
    very thick,
    line join=round,
   ] table [
    x = bins,
    y = hist,
    col sep=comma,
] {Jeffery2_1_5.csv};
\addlegendentryexpanded{1.5 $\tau_\ell$}

  \addplot+[
    color=seq3,
    mark=none,
    solid,
    very thick,
    line join=round,
   ] table [
    x = bins,
    y = hist,
    col sep=comma,
] {Jeffery2_2.csv};
\addlegendentryexpanded{2.0 $\tau_\ell$}

  \addplot+[
    color=seq4,
    mark=none,
    solid,
    very thick,
    line join=round,
   ] table [
    x = bins,
    y = hist,
    col sep=comma,
] {Jeffery225.csv};
\addlegendentryexpanded{2.5 $\tau_\ell$}


  % Plot reference particles
  \path (axis cs: 0,   0.485) coordinate[rec0];  
  \path (axis cs: 45,  0.485) coordinate[rec45];  
  \path (axis cs: -45, 0.485) coordinate[rec45r];  
  \path (axis cs: -90, 0.485) coordinate[rec90];  
  \path (axis cs: 90,  0.485) coordinate[rec90];  

  %\draw[black, <->, very thick, >=stealth] (axis cs: 10, 0.22) -- (axis cs:
  %10, 0.42);
  %\node[black] at (axis cs: 25, 0.325) {$\sim 40\%$};

 \end{axis}

  \begin{axis}[
    xshift=\startx,
    yshift=\starty,
    % clip mode=individual,
    clip=true,
    xmin=-90, xmax=90,
    ymin=0.0,
    ymax=0.51,
    enlargelimits=false,
    axis on top=false,
    legend style={draw=none, 
                  font=\normalsize,
                  fill=none,
                  %row sep=-0.9mm,
    },
    legend columns=4, 
    legend style={row sep=2.75pt},    
    legend plot pos=left,
    %legend pos= south west,
    legend style={at={(axis cs: -86, 0.1)}, anchor=south west},
    legend cell align=left,
    legend style={
                % the /tikz/ prefix is necessary here...
                % otherwise, it might end-up with `/pgfplots/column 2`
                % which is not what we want. compare pgfmanual.pdf
        /tikz/column 2/.style={
            column sep=4pt,
        },
    },
    %reverse legend,
    width = \axiswidth,
    height = \axisheight,
    scale only axis,
    xtick={-90, -45, 0, 45, 90},
    xticklabels={\ang{-90}, \ang{-45}, \ang{0}, \ang{45}, \ang{90}},
    minor ytick={0.1, 0.3, 0.5},
    minor xtick={-75, -60, -30, -15, 15, 30, 60, 75},
    xticklabels={$-\frac{\pi}{2}$,
                 $-\frac{\pi}{4}$,
                 0,
                 $\frac{\pi}{4}$,
                 $\frac{\pi}{2}$}, 
    xtick style={font=\small},
    ytick style={font=\small},
  ]

  \addplot+[
    color=seqb,
    mark=none,
    dotted,
    %solid,
    %only marks,
    %mark size=1pt,
    very thick,
    line cap=round,
    %line join=round,
   ] table [
    x = bins,
    y = hist,
    %y expr = \thisrow{hist} * (180/3.14159265359),
    col sep=comma,
] {meanstddata.csv};
\addlegendentryexpanded{$\left< \theta_p \right>_{\text{Re}, \alpha}$}

  \addplot+[
        draw=none,
        forget plot,
        name path=stdmin,
        mark=none,
        line join=round,
    ] table [
        x = bins,
        y = stdmin,
        %y expr = \thisrow{stdmin} * (180/3.14159265359),
        col sep=comma,
] {meanstddata.csv};


  \addplot+[
        draw=none,
        forget plot,
        name path=stdmax,
        mark=none,
        line join=round,
    ] table [
        x = bins,
        y = stdmax,
        %y expr = \thisrow{stdmax} * (180/3.14159265359),
        col sep=comma,
] {meanstddata.csv};

   \addplot+[
        fill=seqa,
        draw=seqb,
        thin,
        solid,
        opacity=0.3,
        thin,
        line join=round,
   ]
   fill between[
        of = stdmax and stdmin
   ];
\addlegendentryexpanded{$\left< \theta_p \right>_{\text{Re}, \alpha} \pm
\sigma$}
 \end{axis}

% \node at (-0.9, 3.2) {(a)};

\end{tikzpicture} 
\end{document}
