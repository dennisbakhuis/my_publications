\documentclass[tikz,border=5pt]{standalone}
\usepackage{pgfplots}
\usepackage{amsmath}
\usepackage{siunitx}
\usepackage{xcolor}

% original colors
\definecolor{color1}{RGB}{166,118,29}
\definecolor{color2}{RGB}{217,95,2}
\definecolor{color3}{RGB}{231,41,138}
\definecolor{color4}{RGB}{117,112,179}
\definecolor{color5}{RGB}{27,158,119}
\definecolor{color6}{RGB}{102,166,30}
\definecolor{color7}{RGB}{230,171,2}

% darkend first colors
\definecolor{color1a}{RGB}{116,68,0}
\definecolor{color3a}{RGB}{181,0,88}
\definecolor{color4a}{RGB}{67,62,119}
\definecolor{color6a}{RGB}{52,116,0}

% lightend second colors
\definecolor{color1b}{RGB}{216,168,79}
\definecolor{color3b}{RGB}{255,91,188}
\definecolor{color4b}{RGB}{167,162,229}
\definecolor{color6b}{RGB}{152,216,80}

\begin{document}
\begin{tikzpicture}
  [
    rec0/.style={shade,
                rectangle,
                minimum width=6mm,
                minimum height=2.00mm,
                inner sep=0pt,
                outer sep=0pt,
                top color=gray!60!white,
                bottom color=gray!30!white,
                draw=black, 
                very thin},
    rec45/.style={shade,
                rectangle,
                rotate=45,
                minimum width=6mm,
                minimum height=2mm,
                inner sep=0pt,
                outer sep=0pt,
                top color=gray!60!white,
                bottom color=gray!30!white,
                draw=black, 
                very thin},
    rec45r/.style={shade,
                rectangle,
                rotate=-45,
                minimum width=6mm,
                minimum height=2mm,
                inner sep=0pt,
                outer sep=0pt,
                top color=gray!60!white,
                bottom color=gray!30!white,
                draw=black, 
                very thin},
    rec90/.style={shade,
                rectangle,
                rotate=90,
                minimum width=6mm,
                minimum height=2mm,
                inner sep=0pt,
                outer sep=0pt,
                top color=gray!60!white,
                bottom color=gray!30!white,
                draw=black, 
                very thin}
  ]
% width is 8.64cm
\newcommand{\figwidth}{8.64cm}
\newcommand{\figheight}{4.8cm} % Golden ratio

\draw[use as bounding box, white] (0,0) rectangle (\figwidth,\figheight);
  
  \begin{axis}[
    xshift=1.1cm,
    yshift=0.85cm,
    clip mode=individual,
    xmin=-90, xmax=90,
    ymin=0.0,
    ymax=0.65,
    enlargelimits=false,
    axis on top=false,
    xlabel=$\theta_p$,
    ylabel=$\text{PDF}(\theta_p)$,
    legend style={draw=none, 
                  font=\normalsize,
                  fill=none,
                  %row sep=-0.9mm,
    },
    legend plot pos=left,
    legend pos= south west,
    %legend style={at={(axis cs: -80, 0.01)}, anchor=south west},
    legend cell align=left,
    %reverse legend,
    width = 9cm,
    height = 5.50cm,
    ylabel near ticks,
    xtick={-90, -45, 0, 45, 90},
    xticklabels={\ang{-90}, \ang{-45}, \ang{0}, \ang{45}, \ang{90}},
    minor ytick={0.1, 0.3, 0.5},
    minor xtick={-75, -60, -30, -15, 15, 30, 60, 75}
  ]

  \addplot+[
    color=color3,
    mark=none,
    very thick,
    line join=round,
   ] table [
    x = bins,
    %y = hist,
    y expr = \thisrow{hist} * (180/3.14159265359),
    col sep=comma,
] {orientation_mean.csv};
\addlegendentryexpanded{$\left< \theta_p \right>_{\text{Re}, \alpha}$}

  \addplot+[
    color=black,
    mark=none,
    dashed,
    very thick,
    line join=round,
   ] table [
    x = bin,
    y = hist,
    col sep=comma,
] {JefferyAligment.csv};
\addlegendentryexpanded{Jeffery alignment}


  % Plot reference particles
  \path (axis cs: 0,   0.615) coordinate[rec0];  
  \path (axis cs: 45,  0.615) coordinate[rec45];  
  \path (axis cs: -45, 0.615) coordinate[rec45r];  
  \path (axis cs: -90, 0.615) coordinate[rec90];  
  \path (axis cs: 90,  0.615) coordinate[rec90];  

  %\draw[black, <->, very thick, >=stealth] (axis cs: 10, 0.22) -- (axis cs:
  %10, 0.42);
  %\node[black] at (axis cs: 25, 0.325) {$\sim 40\%$};

 \end{axis}

\node at (-0.9, 3.2) {(a)};

\end{tikzpicture} 
\end{document}
