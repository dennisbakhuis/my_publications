% !TEX root = Main document.tex

\section{Summary and conclusions}
The influence of a {hydrophobic} wall on drag reduction was studied in a highly turbulent Taylor--Couette flow. We applied a {hydrophobic} coating to the otherwise smooth and hydrophilic inner cylinder (IC) of the Taylor--Couette setup. In single-phase flow, we found a constant increase in drag of about \perc{14} for the rough hydrophobic wall compared to the smooth hydrophilic wall over the whole range of Reynolds numbers $5.0\e{5} \leq \Rey \leq 1.8\e{6}$ measured. For bubbly two-phase flow however, the addition of air bubbles to the flow resulted in more drag reduction for the rough hydrophobic IC as compared to the hydrophilic IC, using the same volume fraction of air bubbles $\alpha$. For $\alpha \geq 4\%$, more DR was found over nearly the whole range of $\Rey$. Only in the region of $\Rey < 7.0\e{5}$ -- where the measurement uncertainty is highest -- a slight drag increase was found for the {hydrophobic} IC when $\alpha = 6\%$. A strong difference in DR behaviour is found when comparing $\alpha = 2\%$ with $\alpha \geq 4\%$. The void fraction $\alpha = 2\%$ gives a clear drag increase above $\Rey = 10^6$, indicating that the bubble drag mechanism is more effective with a superhydrophic IC when sufficient air bubbles are present in the flow.
This can be explained by the micro-scale surface geometry of the surface, acting as roughness to the flow and hence increasing the drag. The effect of the {hydrophobic} coating is therefore twofold: (i) a more effective bubble drag reduction mechanism, and (ii) an increase in drag from the surface roughness. The role of roughness is confirmed by comparing drag measurements of the {hydrophobic} IC to drag measurements of the smooth hydrophilic IC with $\alpha = 0$, which shows more drag for all values of $\alpha$.

The effect of roughness is more pronounced with larger $\Rey$, since the thickness of the viscous sublayer is then smaller, making it compatible with the roughness length scales. This is confirmed by velocity profile measurements, showing that the normalized azimuthal flow velocities near the {hydrophobic} IC are larger compared to the smooth hydrophilic IC for the same $\Rey$. This indicates that for larger $\Rey$ the {hydrophobic} IC appears rougher for the flow.

Whereas the drag continues to increase with $\Rey$ for the {hydrophobic} IC compared to the hydrophilic IC when $\alpha < 4\%$, the difference in drag appears to level off for $\alpha \geq 4\%$, showing a much weaker dependence on $\Rey$. Apparently, above a certain $\Rey$ with $\alpha \geq 4\%$, the difference between the two competing effects reaches a constant value. The result is a constant increase in DR for the {hydrophobic} IC. The difference in DR does also not vary significantly with $\alpha$ for $\alpha \geq 4\%$. This leads us to the conclusion that, although a minimum amount of air is required for the {hydrophobic} coating to provide more effective bubble DR, adding more air beyond this minimum barrier will not necessarily result in more drag reduction. We hope that our work will give guidelines for industrial applications of bubbly drag reduction in hydrophobic wall-bounded turbulence, such as in naval applications or in pipelines.
