% !TEX root = Main document.tex
\section{Introduction}
Skin friction drag reduction (DR) in turbulent flow is a topic of research that is relevant for many industrial applications. In particular, the maritime industry may benefit from this, since reducing fuel consumption by only a few percent will lead to significant cost savings and reduction of pollutant emission~\citep{vandenBerg2007,Ceccio2010,Murai2014,Park2014,Gose2018}.

In this work we combine {hydrophobic} surfaces with two phase flow to study drag reduction, \add{a combination that, to our best knowledge, has not often been studied before, especially not at the high Reynolds numbers $\Rey$ of up to $1.8\e{6}$ that we reach. The physics behind this combination is interesting, since both hydrophobic surfaces and (air) bubble injection have shown individually to decrease the skin friction drag. At the same time, by increasing the amount of gas in the liquid, the effectivity and life span of a drag reducing superhydrophobic surface can be increased~\citep{Lv2014, Xiang2017}. Compared to a hydrophilic surface, gas bubbles that impact a hydrophobic surface are more likely to attach to the surface and form a lubricating layer~\citep{Kim2017}. Although the wall shear stress in our setup is much larger than what the bubbles in the work of~\citet{Kim2017} are exposed to, a possible result is that the number of bubbles close to the wall increases, which is beneficial for bubbly DR. A set of experiments of two-phase flow over a hydrophobic plate up to $\Rey = 5000$ by~\citet{Kitagawa2019} showed two groups of bubbles. One group of medium-sized free bubbles, and a group of small wall-adhered bubbles, that coalesce into large bubbles. Since the bubbles that stick to the plate change the flow close to the plate, they suggest that the hydrophobic plate is likely to experience more friction drag. Based on this reasoning, they suggest that these results should be carefully considered, when air bubble behaviour is controlled using functionalized (hydrophobic) surfaces in bubbly DR applications~\citet{Kitagawa2019}. Hence, the two methods of drag reduction (bubbly and with hydrophobic surfaces) will influence one-another. However, it is yet unknown whether this is positive or negative for the total combined drag reduction and we want to find this out in this paper.} \\

We explore the difference in skin friction coefficient between two types of surfaces: a very smooth hydrophilic surface and a more rough {hydrophobic} surface. The {hydrophobic} surface is a sheet of porous polypropylene material, commercially available in large quantities. Representative to more practical applications, it has a sponge-like isotropic geometry of distributed (roughness) length scales formed by the porous structure. To study the fully developed turbulence typical for maritime applications, it is desirable to experimentally achieve high Reynolds numbers, and have both the bulk flow and boundary layer in a state of turbulence. To this end, we use the Twente Turbulent Taylor--Couette facility (T$^3$C) described in~\citet{VanGils2011}, of which the inner cylinder is made {hydrophobic} using the porous polypropylene material. This closed system, with an exact energy balance between input (driving of the flow) and output (viscous energy dissipation), allows for accurate measurement of global drag. Due to its excellent optical accessibility, this can be combined with local flow measurements, for instance using particle image velocimetry (PIV), as well as visualisations of the flow structure and the {hydrophobic} surface using (high-speed) imaging techniques. Air bubbles are introduced to the working liquid to demonstrate the drag reducing effect of the {hydrophobic} inner cylinder. This combination of the T$^3$C with a SH inner cylinder and air bubbles in the working fluid, enables us to study {hydrophobic} bubbly drag reduction at industrially relevant high Reynolds numbers in a well controlled condition, giving a better understanding of the mechanisms involved.

The paper is organized as follows: In chapter 2 we give an extensive overview of prior work on bubbly drag reduction, drag on {hydrophobic} surfaces, and drag enhancement of rough walls, as all these effects are crucial to understand the competing effects explored in this paper. In chapter 3 the experimental methods are described. Chapter 4 presents the results and discusses them. The paper ends with conclusions.

\section{Overview over prior work on bubbly drag reduction and (super)hydrohobic surfaces}

\subsection{Drag reduction with {hydrophobic} surfaces}
Superhydrophobic surfaces are typically created by combining a hydrophobic chemistry (resulting in low surface energy) with micro or nanoscale asperities on the surface~\citep{Li2007}. The top of these asperities are in contact with the liquid, while air is captured between the asperities. This effectively reduces the solid-liquid contact area, partially replacing it with a gas-liquid interface, that locally changes the no-slip boundary condition to a shear-free boundary condition.
The gas-liquid interface is supported by the capillary forces, which in general are larger for hydrophobic materials compared to hydrophilic materials of equal geometry. Dependent of chemistry and geometry, a gas-liquid interface can collapse under influence of a pressure or shear force, and transitions into a thermodynamically favoured wetting state. Various types of asperities exist, ranging from structures such as pillars and ridges to pyramids and mushroom-like shapes~\citep{Peters2009,Qi2009,Park2014,Domingues2017}. However, such well-defined shapes are expensive and time-consuming to produce. Therefore, larger areas of SH surfaces~($>\SI{100}{\square\centi\metre}$) usually have a random roughness structure~\citep{Hokmabad2016}. We refer the reader to the review article by~\cite{Li2007} for a broader introduction to SH surfaces.\\

An overview of various experimental and numerical studies in the laminar and low Reynolds number ($\Rey$) turbulent regime is given in the review article by~\cite{Rothstein2010}. Under laminar flow conditions, the behaviour of SH surfaces is typically studied in microchannels. Drag reduction is then quantified by defining a slip length, a slip velocity or by a decrease in pressure drop over the channel~\mbox{\citep{Tsai2009,Haase2013,Park2015}}. As many industrial flows are highly turbulent, it is crucial to study the behaviour of such surfaces in the high Reynolds number flow regime. For marine vessels for example, Reynolds numbers are of the order of~$\Rey = \mathcal{O}(10^9)$.

Superhydrophobic DR in laminar flow only depends on the geometry of the asperities on the surface that set the slip length and determine the slip velocity. For turbulent flows, SH drag reduction also depends on the Reynolds number~\citep{Park2013}. With increasing $\Rey$, the thickness of the viscous sublayer decreases, which is the most relevant length scale when comparing the geometric features of the superhydrophobic surface~\citep{Daniello2009}. In the near-wall region inside the boundary layer of a turbulent flow, the momentum transfer is dominated by molecular interactions, whereas the role of turbulent momentum transfer is negligible. In other words, viscous stress dominates over Reynolds stress. Altering this region affects the entire boundary layer and hence the drag. The outer edge of the viscous sublayer is typically given by a distance $y_\text{vsl} = 5 \nu / u_\tau = 5 \delta_\nu$ from the wall, where $\nu$ is the kinematic viscosity, and $u_\tau = \sqrt{\tau_w / \rho}$ the friction velocity for wall shear stress $\tau_w$ and density $\rho$. The viscous length scale $\delta_\nu = \frac{\nu}{u_\tau}$ is the usual scaling parameter for nondimensionalization to viscous wall units, indicated by a superscript `+', e.g. $y^+ = y/\delta_\nu$.

In laminar flow, DR is a direct result of the shear-free (slip) boundary condition. An additional effect matters in turbulence, where near-wall turbulent structures are suppressed due to the slip boundary condition, resulting in additional DR~\citep{Park2013}. The numerical work of \cite{Park2013} showed that DR increases with the slip length $b^+$, defined as the length below the surface where the extrapolated velocity profile reaches zero. When $b^+ \gtrapprox 30-40$, the drag is not further affected by an increase of $b^+$. This length scale corresponds to the outer edge of the buffer layer $5 < y^+ < 30$~\citep{Pope}, where streamwise near-wall vortical structures primarily reside~\citep{Park2013}. Both observations point in the direction that these near wall structures are very important for the larger DR that is found for turbulent flows over SH surfaces compared to laminar flow over SH surfaces~\citep{Park2013}. \add{In the work of~\citet{Rastegari2018} similar conclusions were drawn. A balance was found between the drag reducing mechanisms of superhydrophobic microgrooves and riblets in the form of a slip velocity together with weakened Reynolds shear stress and near wall vortical structures on the one hand, and a drag increase from the interactions between the microtextures and the flow on the other hand.} Results from experiments by \cite{Daniello2009} suggest a critical Reynolds number that prompts the onset of DR, which corresponds to the transition to turbulent flow. For their system of streamwise-aligned SH ridges in channel flow, they find no DR in the laminar regime, whereas after the flow has transitioned to turbulent flow, significant drag reduction was found. Hence, the physics behind the onset of DR must be related to the structure of the wall-bounded turbulent flow~\citep{Daniello2009}.

We divide the literature on turbulent flow over {hydrophobic} surfaces in two regimes: low (but still turbulent) $\Rey$ turbulence ($\Rey < 10^5$) and high $\Rey$ turbulence ($\Rey > 10^5$). In these regimes, a difference between single-phase and two-phase flow can be made, although most of the research so far has focussed on single-phase flow. Note that in single-phase flow, i.e., when no air is actively added to the working liquid, air might be trapped by the SH surface when the surface is submerged in the working liquid. In two-phase flow, gas is actively dispersed by (for instance air) bubble injection to the working liquid.

Different design rules are suggested in literature for optimal size and spacing of the geometrical features forming the SH surface. In the low $\Rey$ turbulence regime, authors mainly seem to use, or suggest to use, surfaces with pillar/ridge spacing $w^+ > 1$, or with a roughness parameter $k^+>1$. For the high $\Rey$ turbulence regime, however, the opposite is the case: suggested is $w^+ < 1$, or $k^+ < 1$. \add{The study by~\citet{Gose2018} suggests to not only use the normalized roughness $k^+$ to predict the drag reducing properties of a superhydrophobic surface, but to also include the contact angle hysteresis measured at a pressure higher than atmospheric pressure. This is done to simulate the large pressure fluctuations and high shear rates generated by high Reynolds number flows~\citep{Gose2018}. The roughness of the superhydrophic surfaces they studied varied between $k^+ = 0.2$ and $k^+ = 4.5$, with corresponding drag reduction changing from \perc{-90} to \SI{90}{\percent}. Specifically around $\text{DR} = \perc{0}$, the trend of increasing DR with decreasing $k^+$ is absent, showing drag reduction for one surface with $k^+ = 1$ and an increase of drag for another surface with $k^+ < 1$. When $k^+$ was scaled with the roughness parameter and the wetted area fraction, or the high-pressure contact angle hysteresis (\SI{370}{Pa} for a \SI{250}{\nano \litre}), the DR data collapsed to a single curve~\citep{Gose2018}. }

An overview of the different surface parameters found in literature focussing on DR with SH surfaces is shown in table~\ref{table:DRtable}. \\
%\include{DRtable}
\begin{sidewaystable}
\begin{tabularx}{\textwidth}{lllllll}
\hline
Author & Phase & Flow type & SH surface & \Rey & DR$_\text{max}$ & Surface roughness\\
\cite{Srinivasan2011} & single & Taylor--Couette & random & $1.6\times10^3-8.0\times10^4$ & \SI{22}{\%} & \makecell[tl]{$k^+ = 1.4$\parnote[a]{derived from data in paper}\\ $w^+ = 2.5$\parnotemark{a} } \\
\cite{Rastegari2018} & single & channel (LB) & longitudinal grooves & $3.6\e{6}$ & \SI{61.1}{\%} & $w^+=8$\parnote[b]{normalized with value for no-slip surface} \\
\cite{Daniello2009} & single & channel & ridges & ${3.0}\times10^3-{6.0}\times10^3$ & \SI{50}{\%} & $w^+ > 1$\parnote[c]{optimal value obtained from parameter sweep} \\
\cite{Martell2009}   & single & channel (DNS) & \makecell[tl]{ridges \\ posts} & $4.2\e{3}$\parnotemark{a} & \SI{40}{\%} & \makecell[tl]{$w^+$ as large \\ as possible} \\
\cite{Rosenberg2016} & single & Taylor--Couette & triangular ridges & $6.0\e{3}-9.0\e{3}$ & \SI{10}{\%} & \makecell[tl]{$k^+ = 3.9$ \\ $w^+ = 5.6$ } \\
\cite{vanBuren2017} & single & Taylor--Couette & ridges & $6.0\e{3}-1.0\e{4}$ & \SI{9}{\%} & $w^+ = 35$\\
\cite{Park2013} & single & channel (DNS) & ridges & $4.2\e{3} - 1.7\e{4}$ & \SI{90}{\%} & $w^+ = 100$\parnotemark{c} \\
\cite{Gose2018} & single & channel & random & $1.0\e{4}-3.0\e{4}$ & \SI{90}{\%} & $k^+<0.5$\parnotemark{c}\parnote[d]{additional surface characterization} \\
\cite{Panchanathan2018} & single & Taylor--Couette & square pillars & $4.7\e{4}$ & \SI{3}{\%} & \makecell[tl]{$k^+ = 13.6$\parnotemark{a} \\ $w^+ = 27.1$\parnotemark{a} }  \\
\hline
\cite{Du2017} & dual & channel & random & $1.2\e{5}$ & \SI{20}{\%} & not enough info given \\
\cite{Fukuda2000} & dual & \makecell[tl]{channel \\ flat plate \\ ship model} & random & \makecell[tl]{$5.0\e{4}-4.0\e{5}$ \\ $3.0\e{5}-1.7\e{7}$ \\ $9.0\e{5}-8.0\e{8}$} & \makecell[tl]{\SI{50}{\%}\parnote[e]{decreasing with \Rey} \\ \SI{50}{\%}\parnotemark{e} \\ \SI{20}{\%} } & no info given \\
\cite{Aljallis2013} & single & flat plate & random & $3.0\e{5}-3.0\e{6}$ & \SI{30}{\%}\parnote[f]{no DR for $\Rey_\text{L} > 10^6$} & no info given \\
\cite{Ling2016} & single & channel & random + ridges & $1.0\e{5}$  & \SI{36}{\%} & $k^+ = 0.68$\parnotemark{c} \\
\cite{Park2014} & single & channel & ridges & $1.0\e{5}-1.0\e{6}$ & \SI{75}{\%} & $w^+<1$\parnotemark{c} \\
\cite{Reholon2018} & single & cylindrical object & random &  $5.0\e{5} - 1.5\e{6}$ & \SI{36}{\%}\parnotemark{e} & $k^+ = 0.5$ \\
\cite{Bidkar2014} & single & channel & random & $1.0\e{6}-9.0\e{6}$ & \SI{30}{\%} & $k^+<0.5$\parnotemark{c} \\

\hline
\end{tabularx}
\parnotes
\caption{Overview of literature on drag reduction (DR) of turbulent flows over superhydrophobic (SH) surfaces, illustrating different surface design parameters $k^+$ and $w^+$ corresponding to the largest drag reduction found by different authors. The horizontal line seperates the \textit{low $\Rey$ turbulence} from the \textit{high $\Rey$ turbulence} as introduced in section 2. \label{table:DRtable}}
\end{sidewaystable}

\subsubsection*{Low $\Rey$ turbulence}
Using channel flow, \cite{Daniello2009} studied a variety of SH surfaces consisting of streamwise aligned ridges, with varying ridge spacing $w^+ = \numrange{1}{4}$. Over the whole range of $3 \times 10^3 \leq \Rey \leq 6 \times 10^3$, a DR of \SI{50}{\percent} was found~\citep{Daniello2009}. The dependence of DR on surface feature size has also been studied using Direct Numerical Simulations (DNS) by \cite{Martell2009}, finding good agreement to the work of \cite{Daniello2009}. More recent DNS of streamwise SH ridges in channel flow by \cite{Park2013}, showed a maximum DR when the ridge spacing was similar to the spacing between near-wall turbulent structures $w^+ = 100$. The work of \cite{Park2013} was able to isolate the effect of the SH surface, since it was modelled as a flat surface with an alternating no-slip and no-shear boundary condition. Effects of roughness on the flow that would play a role in experiments, either from a non-flat gas-liquid interface or from surface features that protrude through the viscous sublayer, could therefore be ruled out.

Rather than a surface of well defined feature size and geometry, a porous surface of random roughness structure was used by~\cite{Srinivasan2011}. The inner cylinder of their Taylor--Couette was was made superhydrophobic by spraycoating a mixture of PMMA fibres and low surface energy fluorodecyl POSS molecules. Nonetheless are the resulting surface roughness parameters similar to that of~\cite{Daniello2009}. From the work of \cite{Srinivasan2015} we calculate the average roughness height at the maximum $\Rey = 8\times 10^4$ to be about $k^+ = 1.5$ and the mean roughness spacing $w^+ = 2.5$. The maximum $\Rey$ also resulted in the largest DR of $\SI{22}{\percent}$. Another study in Taylor--Couette, of similar \Rey, but with much larger surface roughness parameters of $k^+ = 27$ and $w^+ = 14$ formed by a SH pillar structure, found only \SI{3}{\percent} DR~\citep{Panchanathan2018}. When instead of large SH pillars, large streamwise aligned SH ridges were used in Taylor--Couette, an optimal groove spacing of $w^+ = 35$ was found for achieving a maximum DR of $\SI{35}{\percent}$~\citep{vanBuren2017}. For the smallest groove spacing tested, $w^+ = 2$, no DR was found. The base line drag used in the definition of the drag reduction is very important. Where \cite{vanBuren2017} used their wetted surface as the baseline, was a smooth surface used for the baseline drag by~\cite{Panchanathan2018}. When the baselines are defined equally, the difference in DR found between both studies will be much smaller.

\subsubsection*{High $\Rey$ turbulence}
\cite{Ling2016} measured the velocity in the inner part of the turbulent boundary layers over SH surfaces subjected to single-phase flow. Surfaces were made SH by means of spraycoating, resulting in a random oriented roughness, and by etching and coating, giving both ridges and random oriented roughness. Measurements were done in a water tunnel, operated at $1 \times 10^5 \leq \Rey \leq 3 \times 10^5$. Their results revealed a delicate balance between the contribution of viscous stresses and Reynolds stresses to the wall shear stress. This balance determines whether DR is found (viscosity dominates), or the surface roughness increases the drag (turbulence dominates). It was found that when the roughness $k^+ \gtrapprox 1$, the Reynolds stresses become the main contributor to the wall shear stress, and less DR was found~\citep{Ling2016}.

The number of studies we found that combine a SH surface and air injection (two-phase flow) is limited. \cite{Du2017} only found DR when air was being injected through their SH surface. The DR was the result of weakened near-wall vortices, pushed away from the SH surface, and smaller shear rates on top of the SH surface~\citep{Du2017}. A variety of flow geometries was studied by~\cite{Fukuda2000}: rectangular pipe flow ($5\times 10^4 \leq \Rey \leq 4 \times 10^5$), flat plate ($3\times 10^5 \leq \Rey_\text{L} \leq 1.7 \times 10^7$), and ship models in a towing tank ($9\times 10^5 \leq \Rey_\text{L} \leq 8 \times 10^8$). For the pipe flow and flat plate experiments, the maximum DR of \SI{50}{\percent} was found to decrease with $\Rey$ to $\sim \SI{0}{\percent}$. Negligible influence of an increased air injection rate on DR was observed for all flow geometries~\citep{Fukuda2000}.

One of the few experiments in the high $\Rey$ turbulence regime that uses a surface with a geometrically well defined pattern is done by \cite{Park2014} ($1\times 10^5 \leq \Rey \leq 1 \times 10^6$), allowing for a direct comparison to the work of \cite{Daniello2009} ($3\times 10^3 \leq \Rey \leq 6 \times 10^3$) in the low $\Rey$ turbulence. Both studies made use of a fully turbulent, single-phase channel flow  over a surface of streamwise SH ridges. \cite{Daniello2009} suggested an optimum ridge spacing of $w^+ = 5$, which is equal to the size of the viscous sublayer. \cite{Park2014} however, found their maximum DR for $w^+ < 1$. This is a difference typically found between studies in the low- and the high $\Rey$ turbulence regime, as can also be seen in table~\ref{table:DRtable}.

\subsection{The air plastron}
The air layer captured between the SH surface and the water is commonly referred to as the air plastron. When the SH surface transits from a non-wetted Cassie-Baxter state to a wetted Wenzel state, the plastron and the DR are lost. Since the Wenzel state is typically the thermodynamically more favoured state, it is therefore crucial to prevent or delay this transition. This can for instance be achieved by reducing the size (diameter or $w^+$) of the asperities in which the gas is trapped to increase the Laplace pressure, or by increasing the hydrophobicity of the surface. The diffusion of gas from the plastron into the liquid is another factor to minimize in order to sustain DR, which can for instance be achieved by increasing the amount of saturated gas in the liquid~\citep{Lv2014,Xiang2017}.

In the experiments by \cite{Srinivasan2015}, the SH surface was not fully submerged, resulting in a connection between the plastron and the air present in the room. More DR (\SI{22}{\percent}) was found compared to the case where the air-layer is isolated ($\text{DR} = \SI{15}{\percent}$) for the same $\Rey$~\citep{Srinivasan2015}. When the surface is exposed to flow, the loss of plastron volume can be described by a convection-diffusion mechanism. Larger flow velocities give shorter effective diffusion lengths, resulting in an accelerated transport of gas from the plastron into the the liquid~\citep{Xiang2016}. Video recordings of the plastron exposed to turbulent flow ($5.0 \e{5} \leq \Rey_L \leq 1.5\e{6}$) showed constant movement and variations in the thickness of the plastron, caused by pressure fluctuations in the turbulent boundary layer~\citep{Reholon2018}. \cite{Du2017} found DR when injecting air through a pinhole in their SH surface. The amount of injected air was not enough to form an air bubbly flow, but enough to maintain a plastron that was thick enough to prevent the surface roughness features from contacting the liquid. When the air injection was stopped, the air plastron became thinner, and roughness effects started to play a role~\citep{Du2017}. When the roughness elements are exposed to the flow, the Reynolds stresses become the main contributor to the wall shear stress, resulting in less DR~\citep{Ling2016}.

\add{For this particular reason,~\citet{Gose2018} suggested to measure the surface characteristic contact angle hysteresis under higher than ambient pressures.} Also mechanical interactions between the plastron and solid pollutants in the liquid phase can decrease the plastron stability. Collisions between particles added to the flow and the plastron were shown to shorten its lifetime by about $\SI{50}{\%}$~\citep{Hokmabad2017}. Once the air plastron is destroyed and the surface has transited to the wetted state, energy is required to reverse the surface to the un-wetted state. Different studies explored for instance film boiling, water splitting by electrolysis and the injection of air bubbles into the boundary layer (dual-phase flow) to achieve this~\citep{Saranadhi2016,Panchanathan2018}.

\subsection{Bubbly drag reduction}
\add{The introduction of air bubbles to a flow can also result in reduced skin friction drag.} A typical approach is to inject air bubbles close to (or in) the boundary layer.
We refer to the review articles by~\cite{Ceccio2010} and \cite{Murai2014} for an overview of various studies on bubbly DR.
Early air-lubrication DR experiments, in which gas micro-bubbles were injected (or created) in the (turbulent) boundary layer, showed an increase in DR with increasing air injection rate, and a decrease in DR with increasing Reynolds number~\citep{McCormick1973,Madavan1985,Watanabe1998}. Up to \SI{80}{\percent} reduction of skin friction drag using microbubble injection was reported~\citep{Madavan1984}. This DR was attributed to a thickening of the viscous sublayer (so a smaller gradient in the velocity profile near the wall) caused by the microbubbles that were present in the near-wall buffer layer~\citep{Ceccio2010}.

For Taylor--Couette flow, for low $\Rey$ and microbubble injection, the drag reduction was shown to be due to the weakening or even destruction of the Taylor-rolls, due to the rising microbubbles. This gravity effect (controlled by the Froude number) indeed decreases with increasing Reynolds numbers~\citep{Sugiyama2008,Lohse2018}. More recent research showed the influence of the bubble size on DR, concluding that the existence of large, deformable bubbles, i.e. those that have a large Weber number, is crucial for drag reduction in high $\Rey$ turbulent flows~\citep{Lu2005,vandenBerg2005,vanGils2013,Verschoof2016,Spandan2018}. In these papers, the increase of the Weber number with increasing $\Rey$ is used to explain the enhanced bubbly DR that is typically found for larger $\Rey$~\citep{vandenBerg2005}.

Although the principle of air bubbly DR is not yet fully understood, it is clear that the effect is largest when the bubbles are close to, or in, the boundary layer. For flat plate experiments, the skin friction bubbly drag reduction is commonly limited to the first few metres downstream of the air injector~\citep{Watanabe1998,Sanders2006}. Further downstream, turbulent diffusion causes bubbles to move away from the wall~\citep{Murai2014}. A similar mechanism was observed in Taylor--Couette flow, where strong secondary flows transport bubbles away from the inner cylinder, resulting in a decrease of DR~\citep{vandenBerg2007,Fokoua2015,Verschoof2018}.  \\


\subsection{Roughness}
To create a SH surface, some form of roughness has to be introduced on the surface, to facilitate an asperity where air can be trapped. Since any form of surface roughness increases the drag on a wall-bound flow, we therefore deal with opposing effects in drag reduction using SH surfaces: drag reduction due to air (plastron) lubrication and drag increase from the added roughness. We refer to the reviews of \cite{Jimenez2004} and \cite{Flack2010} for a comprehensive overview of studies towards the influence of roughness on turbulent flows.

Three different roughness regimes are distinguished. In the hydrodynamically smooth regime, when the equivalent sand roughness is less than the thickness of the viscous sublayer ($k^+ < 5$), the surface can be regarded as smooth~\citep{Schlichting}. The perturbations in the flow that are generated by the roughness features of the surface are completely damped out by the viscosity~\citep{Flack2014}. When the roughness $k^+$ increases, parts of it will extend through the viscous sublayer, corresponding to the transitionally rough regime ($5 \leq k^+ < 70$). The log-law that describes the velocity profile close to the wall shifts inwards, maintaining its shape, but reduced in magnitude. The mean velocity profile in the bulk of the flow however stays unaffected by the roughness~\citep{Flack2014}. Hence, universality is only seen for the larger length scales of the flows~\citep{Pope}.

The wall shear stress in the transitionally rough regime is composed of a combination of viscosity and pressure drag on the roughness elements. With increasing roughness height, the contribution of pressure drag increases~\citep{Verschoof2018EPJE}. In the fully rough regime ($k^+ \geq 70$), the pressure drag heavily dominates over viscosity. As a result, the shift in the log law (the roughness function $\Delta U^+$), scales linearly with $k^+$, and the skin-friction coefficient becomes independent of $\Rey$~\citep{Flack2014}.

Although the size of the roughness $k^+$ gives a good indication for the state of roughness: hydrodynamically smooth, transitionally rough or fully rough, which also depends on the geometry of the roughness. For instance, a stepwise geometry that consists of steep slopes will transit to the fully rough regime at smaller $k^+$ than a roughness of more gentle slopes~\citep{Busse2017}. Similarly, a surface of very closely packed roughness elements (high solidity), or a surface where the roughness elements are sparse (high porosity) will behave more like a surface of smaller $k^+$~\citep{MacDonald2016}.
In the context of Taylor--Couette turbulence, roughness effects were analyzed by \cite{Zhu2018} and \cite{Berghout2018}, who found the same universal $\Delta U^+ \left( k^+ \right)$ for the velocity reduction as was found by \cite{Nikuradse1933} for pipe flow.
