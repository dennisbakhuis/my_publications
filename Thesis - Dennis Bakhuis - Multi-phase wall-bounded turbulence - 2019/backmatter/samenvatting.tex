\chapter{Summary (Dutch)}
\chaptermark{Summary (Dutch)}
\markboth{\MakeUppercase{Summary (Dutch)}}{}
In vele geofysische systemen en in bijna alle industri\"ele toepassingen
zijn turbulente stromingen begrensd door wanden.
Deze stromingen bestaan vaak uit meerdere fasen, dat wil zeggen stromingen met
\'e\'en of meer insluitsels zoals bellen, druppels, of deeltjes.
Het huidige begrip over de fysica van dit soort stromingen is nog steeds beperkt
en dit maakt het belangrijk om deze te bestuderen.
In dit proefschrift worden deze wandbegrensde meerfasenstromingen bestudeerd in
twee canonieke systemen: Taylor-Couette-stroming en
Rayleigh-B\'enard-convectie.\\
% chap 1
\indent Door het toevoegen van stijve neutrale bolvormige deeltjes aan de stroming,
wilden wij onderzoeken of deze een vergelijkbare weerstandsvermindering geven
als luchtbellen.
Onze globale torsiemetingen toonden aan dat deze deeltjes de weerstand
nauwelijks veranderden, zelfs bij gebruik van zeer grote volumefracties.
Wij hadden verwacht dat het toevoegen van deze deeltjes de effectieve
viscositeit verhoogt, met als gevolg een stijging van de weerstand.
Deze toename werd echter niet gevonden.
Wij vermoedden dat het weerstandverminderende effect het effect van de
verhoogde viscositeit min of meer opheft.\\
% chap 2
\indent In werkelijkheid kunnen luchtbellen bijna elke vorm aannemen, een eigenschap
die mogelijk uiterst belangrijk is voor het weerstandsverminderende effect.
Om deze hypothese te testen, hebben we cilindrische neutrale deeltjes
ge\"introduceerd in ons systeem.
Hoewel wij nauwelijks verschil in weerstand zagen tussen de bolvormige en
de cylindrische deeltjes, hebben wij ontdekt dat deze deeltjes een
voorkeursori\"entatie hebben ten opzichte van de binnencilinder.
Dit was zeer onverwacht, omdat vaak wordt verondersteld dat een systematische
ori\"entatie in uiterst turbulente stromingen onmogelijk is.
Door de ori\"entatie van de deeltjes te modelleren met behulp van de
Jeffery-vergelijkingen kunnen we de distributie van de ori\"entatie
redelijkerwijs voorspellen.\\
% chap 3
\indent Door gebruik van niet-mengbare vloeistoffen zijn we in staat om insluitsels te
cre\"eren die elke vorm kunnen aannemen.
Hiervoor gebruikten wij water en siliconenolie, waarbij de siliconenolie
een viscositeit heeft die vergelijkbaar is met die van het water.
We hebben de verhouding tussen deze twee vloeistoffen gevarieerd
tussen 0 en \SI{100}{\percent}.
De gecre\"eerde emulsies kunnen twee verschillende vormen hebben.
Door olie toe te voegen aan water ontstaan oliedruppels in water.
De  effectieve viscositeit van het mengel wordt hoger dan de viscositeiten van
de individuele fasen.
Bij overschrijden van een kritische hoeveelheid olie, vindt een inversie plaats
en ontstaat een situatie van waterdruppels in olie.
Voor deze vorm treedt ook weerstandsvermindering op en we denken dat dit in gang
wordt gezet doordat de waterdruppels zeer groot zijn, in vergelijking met de
oliedruppels.
Door een microscoop op het syteem te bouwen konden we deze hypothese
bevestigen en zagen wij dat de waterdruppels 14 keer groter zijn dan de
oliedruppels.\\
% chap 4
\indent Om te onderzoeken wat het effect is van niet-ideale grenscondities
maakten wij gebruik van een ander canoniek systeem: Rayleigh-B\'enardconvectie.
Dit deden we door middel van directe numerieke simulaties, waarbij we de
bovenmuur hebben verdeeld in banen van geleidend en isolerend materiaal. 
De breedte en hoeveelheid van deze banen werden gevarieerd, terwijl de
oppervlakte van het isolerende en geleidende gedeelte aan elkaar gelijk
bleven.
Voor een bovenmuur met de grootste hoeveelheid banen is het warmtetransport
bijna identiek aan een systeem met een volledig geleidende bovenmuur.
Dit betekent dat kleine inperfecties van de temperatuurgrenscondities niet
cruciaal zijn voor de algemene stroming in het systeem.\\
% chap 5
\indent We kunnen ook niet-ideale grenscondities cre\"eren in het
Taylor-Couettesysteem, door de binnencilinder gedeeltelijk ruw te maken met
schuurpapier.  Door de periodiciteit van ruwe en gladde vlakken te variëren,
waren we in staat om de secundaire stromingen te besturen.  Bij gebruik van de
grootste ruwe vlakken zijn er slechts twee individuele rollen, in
tegenstelling tot de optimale breedte waarbij 12 rollen zichtbaar zijn.  Het
draaimomenttransport heeft een optimum als de breedte van de ruwheid gelijk is
aan de afstand tussen beide cilinders.  Hiermee hebben wij aangetoond dat een
klein verschil in configuratie van de ruwheidselementen de structuur van de
stroming enorm kan wijzigen.
