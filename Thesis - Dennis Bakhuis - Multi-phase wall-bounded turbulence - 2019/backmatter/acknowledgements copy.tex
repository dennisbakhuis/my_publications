%%%%%%%%%%%%%%%
%%%% Flags %%%%
%%%%%%%%%%%%%%%
% % Dutch flag
% \definecolor{flagnlred}{RGB}{191, 57, 34}
% \definecolor{flagnlblue}{RGB}{25, 49, 83}
% \DeclareRobustCommand\flagnl[1]{%
%     \tikz[
%       scale=#1,
%     ]{%
%         \fill[flagnlred] (-3mm, 2mm) rectangle (3mm, 0.666mm);%
%         \fill[white] (-3mm, 0.666mm) rectangle (3mm, -0.666mm);%
%         \fill[flagnlblue] (-3mm, -0.666mm) rectangle (3mm, -2mm);%
%     }%
% }%
% % American flag
% \definecolor{Cardinal}{HTML}{C41E3A}
% \definecolor{Sapphire}{HTML}{012B7F}
% \def\usflagheight{4}
% \def\usflagwidth{1.9 * \usflagheight}
% \def\usflagstripe{0.077 * \usflagheight}
% \tikzstyle{usstars}=[star, star point ratio=2.61, minimum size=0.8 * \usflagstripe mm, fill=White]
% \DeclareRobustCommand\flagus[1]{%
%     \tikz[
%       scale=#1,
%     ]{%
%         % Stripes
% 	    \foreach \x [evaluate=\i as \y using int(2*\x+1)] in {0, ..., 6}{
% 	    	\fill[Cardinal] (0 mm, 2*\x*\usflagstripe mm) rectangle
%                (\usflagwidth mm,\y*\usflagstripe mm);
% 	    }
%         % blue box
% 	        \fill[Sapphire] (0, 6*\usflagstripe mm) rectangle
%                (0.4*\usflagwidth mm, \usflagheight mm);
% 	    % Stars
% 	    \foreach \y in {1, 3, ..., 9}{ \foreach \x in {1, 3, ..., 11}{
% 	      \node[usstars, scale=#1 * 0.02] at (0.033*\usflagwidth*\x mm,
%              6*\usflagstripe mm + 0.054*\usflagheight*\y mm){};
% 	    }}
% 	    \foreach \y in {1, ..., 4}{ \foreach \x in {1, ..., 5}{
% 	      \node[usstars, scale=#1 * 0.02] at (0.066*\usflagwidth*\x mm,
%              6*\usflagstripe mm + 0.108*\usflagheight*\y mm){};
% 	    }}
%     }%
% }%
% % German flag
% \definecolor{flagdered}{RGB}{255, 0, 0}
% \definecolor{flagdegold}{RGB}{255, 204, 0}
% \DeclareRobustCommand\flagde[1]{%
%     \tikz[
%       scale=#1,
%     ]{%
%         \fill[black] (-3.33mm, 2mm) rectangle (3.33mm, 0.666mm);%
%         \fill[flagdered] (-3.33mm, 0.666mm) rectangle (3.33mm, -0.666mm);%
%         \fill[flagdegold] (-3.33mm, -0.666mm) rectangle (3.33mm, -2mm);%
%     }%
% }%
% % China flag
% \definecolor{flagcnred}{RGB}{255, 0, 0}
% \definecolor{flagcngold}{RGB}{255, 255, 0}
% \tikzstyle{cnstars}=[star, star point ratio=2.61,
%                      minimum size=0.2 * 6mm, fill=flagcngold]
% \DeclareRobustCommand\flagcn[1]{%
%     \tikz[
%       scale=#1,
%     ]{%
%         \fill[flagcnred] (-3mm, 2mm) rectangle (3mm, -2mm);%
%         \node[cnstars, scale=#1 * 0.13] at (-3mm + 0.1667 * 6mm,
%             2mm - 0.25 * 4mm){};
%         \node[cnstars, scale=#1 * 0.045, rotate=45] at 
%             (-3mm + 0.3333 * 6mm, 2mm - 0.10 * 4mm){};
%         \node[cnstars, scale=#1 * 0.045, rotate=-45] at 
%             (-3mm + 0.40 * 6mm, 2mm - 0.20 * 4mm){};
%         \node[cnstars, scale=#1 * 0.045, rotate=0] at 
%             (-3mm + 0.40 * 6mm, 2mm - 0.35 * 4mm){};
%         \node[cnstars, scale=#1 * 0.045, rotate=45] at 
%             (-3mm + 0.3333 * 6mm, 2mm - 0.45 * 4mm){};
%     }%
% }%
% german flag
\DeclareRobustCommand\flagde[1]{%
    \tikz[
    ]{%
        \node[inner sep=0pt] at (0,0)
            {\includegraphics[width=#1]{germany.png}};
    }%
}%
% china flag
\DeclareRobustCommand\flagcn[1]{%
    \tikz[
    ]{%
        \node[inner sep=0pt] at (0,0)
            {\includegraphics[width=#1]{china.png}};
    }%
}%
% dutch flag
\DeclareRobustCommand\flagnl[1]{%
    \tikz[
    ]{%
        \node[inner sep=0pt] at (0,0)
            {\includegraphics[width=#1]{nederland.png}};
    }%
}%
% Spanish flag
\DeclareRobustCommand\flagsp[1]{%
    \tikz[
    ]{%
        \node[inner sep=0pt] at (0,0)
            {\includegraphics[width=#1]{spain.png}};
    }%
}%
% usa flag
\DeclareRobustCommand\flagus[1]{%
    \tikz[
    ]{%
        \node[inner sep=0pt] at (0,0)
            {\includegraphics[width=#1]{usa.png}};
    }%
}%
% margin command
\newcommand\flagsize{5mm}
\newcommand\mf[1]{%
    \marginnote{#1{\flagsize}}%
}%
\newcommand\mfl[1]{%
    \marginnote[\hspace*{-1.4cm} #1{\flagsize}]{}%
    % \hspace{1.2cm}%
}%
%%%%%%%%%%%%%%%%%%%%%%%%%%
%%%% Acknowledgements %%%%
%%%%%%%%%%%%%%%%%%%%%%%%%%
\chapter{Acknowledgements}
% General
\mf{\flagus}\indent In the years I have read many acknowledgements from other
theses, however, writing your own acknowledgements for your own thesis is a
strange sensation.  It makes you realize that a four year period that seemed
so long in the beginning is now ending and what an unbelievable journey it has
been.  The Physics of Fluids group is an unique environment which has the
property to cluster and attract amazing people.  It always felt as a privilege
to be part of this group, from which I have benefited tremendously.  I am
thankful to the many people I met or worked with over the years, and I do my
best to mention them all. \\
% Chao
\mf{\flagus}\indent Dear Chao, I still remember the day that we first met. I
was doing my internship together with Julian and Vivek and you entered the
water tunnel lab in your black \emph{Nike Air Max} shoes.  A few years later,
you asked me for this project and I think this was one of the best choices of
my life. I was able to learn from one of the best experimentalists in the
world and your enthousiastic guidance helped me to get where I am now. What I
really appreciated was that you do not hesitate to call from Beijing when I
was in need of a pep talk. Also the peking duck we had together in Beijing
showed your generous hospitality. \mf{\flagcn} 谢谢 for being a brilliant
supervisor and a great friend.\\
% Detlef
\mf{\flagde}\indent Detlef, we both speak fluent Dutch and German, still we
generally end up speaking English. Therefore, I think this part should be in
German.\\
% Sander
\mf{\flagnl}\indent Beste Sander, ik hoorde van jou bestaan door een
schaterend gelach door de gangen van Meander. Een aantal maanden later zaten
we samen hoog volumineus te lachen terwijl jij mij de fijne kneepjes van de
Taylor-Couette liet zien. Ons gevoel voor humor ligt duidelijk op \'e\'en lijn
en er zijn alvast twee eersterangs plaatsen op een specifieke plek gereserveerd.
Jouw bijdrage aan mijn onderzoek was onmisbaar en ik heb het gevoel dat sinds
jij terug bent in de groep, alles in een stroomversnelling is geraakt.
Bijna dagelijks kwam ik bij jouw kantoor met kleine vragen of om gewoon de
snoepla te plunderen. Bedankt voor al je tijd, raad, en snoep door de jaren
heen en ik kijk uit naar de komende projecten die wij nog samen gaan doen.\\
% work committee
\mfl{\flagnl}Beste Tom, Bert, Peter, Leo, en Ren\'e, dit
project is in samenwerking met verschillende industri\"ele partners en
hiervoor hadden wij jaarlijks een gezamelijke bijeenkomst. Deze heb ik altijd
als zeer prettig en nuttig ervaren.\\
% Jacco Snoeier & Laura Stricker
\mfl{\flagus}Dear Jacco, dear Laura, during my pre-master program I
followed the course \emph{Physics of fluids}, which was amazingly lectured by
Jacco and ta'ed by Laura. This made me actually choose for the fluids track.\\
% Rodolfo
\mfl{\flagus}Dear Rodolfo, it was always fun to work with you and now,
after having multiple students myself, admire your patients. The numerous
``how is the manuscript going'' are still haunting me in my sleep. In the end
we got the manuscript published and the whole process helped me a lot with all
the articles I wrote later. I have good memories of my first and only
thanksgiving in Boston. We still need to go diving together!\\
% Roberto Verzicco
\mfl{\flagus}Dear Roberto, thank you for all the help during the writing of my
first manuscript, together with Rodolfo.\\
% Lab: Ruben
\mfl{\flagnl}Beste Ruben, ik heb altijd graag met jou samen gewerkt.
Tijdens het sleutelen hadden we altijd uitgebreide maatschappelijke
discussies, en wat moesten we toch vaak sleutelen. Ook leerde je mij om soms
iets eenvoudiger te denken, hierdoor gingen de projecten later ook een stuk
vlotter. Weet jij trouwens waar de imbus-acht is gebleven?\\
% lab: Rodrigo
\mfl{\flagus}Dear Rodrigo, first I would like to appologize for all the
Mexican stereotypical jokes. Next, I want to thank you for all your effort as
the lab has never been so clean in years. Jokes aside, it was always a great
pleasure to work with you. You are always very careful and precise, which is a
valueable addition to my sometimes blunt approaches. We shared the nightshift
during our MSc., later the night shift at the end of our PhD, and finally we
will also share the night shift of our graduation. One thing\\
\mfl{\flagsp}\hspace{-0.6cm} you should know is that that day \emph{soy un pinguino.}\\
% lab: Pim
\mfl{\flagnl}Beste Pim, wij hebben heel wat lagen op de TC geplakt en weer
verwijderd. Tijdens meerdere bijeenkomsten hebben wij gediscussieerd over onze
uitkomsten, tenminste als je niet naar de kapper moest. Bedankt voor alle hulp
en waarschijnlijk werken we nog even samen.\\
% lab: Roeland + students
\mfl{\flagnl}Dear Roeland, dank je wel voor je hulp aan het begin van mijn
promotie.\\
\mfl{\flagus}Dear Valentin, Charles, Michah, Raymond, and Dominic, I had big
fun working with you guys on the various projects. We have achieved great
things, including two publications.\\
% Alvaro Marin


% Dennis v. Gils

% fellow adventurers

% Erwin van der Poel

% Matthijs Neut


