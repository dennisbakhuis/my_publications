\chapter*{Conclusions}\label{chap:Conclusions}
\graphicspath{{fig/}}
\chaptermark{Conclusions}
\markboth{\MakeUppercase{Conclusions}}{}
\addcontentsline{toc}{chapter}{Conclusions}
In this thesis we investigated turbulent flows with inclusions, \textit{i.e.}
a flow with bubbles, droplets, or solids dispersed in the continuous phase,
and the influence of inhomogeneous boundaries on a turbulent flow.
This was done experimentally using a Taylor--Couette apparatus
(chapters\,\ref{chap:spheres}--\ref{chap:emulsions} and \ref{chap:spanwise})
and using direct numerical simulations of Taylor--Couette flow and
Rayleigh-B\'enard convection (chapters\,\ref{chap:mixedbc} and
\ref{chap:spanwise}).
The findings of each chapter are discussed below.\\
% Spheres
\indent In \refc{chap:spheres} we have performed torque measurements for
flows containing rigid spherical neutrally buoyant particles.
Using particles of various sizes and quantities, we isolated the effect of
size and volume fraction on the drag of a rotating cylinder.
We showed that, unlike bubbles, rigid particles barely alter the drag of the
system, even for cases where their size was comparable to that of the bubbles.
For the sizes 1.5--\SI{8}{\milli\metre}, tested in this thesis, the size
effect was marginal, with a slightly larger effect for the smallest particles.
When increasing the volume fraction of the particles, there is a modest change
in drag, but smaller than can be explained with the apparent viscosity of the
suspension.
The particles cannot be perfectly matched with the density of fluid and
therefore, density effects might apply.
By marginal variations in density of the working fluid, we found that there
was a small but noticeable trend towards drag reduction for lower values of
the particle to fluid density ration $\phi$.
This suggests that a low density of the particle could be a necessary
ingredient for drag reduction. 
The local flow was probed using laser Doppler anemometry at the center of the
gap.
With addition of the particles, the fluctuations of the fluid flow were enhanced,
with wider tails of the velocity probability distributions.
This is generally seen when the relative velocity of the fluid and the
particle is large, which is plausible due to the inertia of the particle.
These fluctuations increase with decreasing particle size or increasing volume
fraction, especially when we measure closer to the inner cylinder.\\
\indent In the first chapter the size of the particles changed while the shape
was fixed to a sphere.
To investigate the shape effect of particles, in \refc{chap:fibers} we used
fiber-like particles.
Due to the anisotropic shape, the underlying flow is influenced by the
orientation of the fiber, and vice versa, the orientation and translation of
the fiber is also affected by the flow.
By using high-speed imaging, we captured both, the orientation and position of
the particle, as function of time.
These fibers, despite their large size, follow the flow almost like
faithful tracers.
We explain this by comparing the turbulent dynamic time at the scale of the
fiber length to the non-linear drag at the finite Reynolds number of the
fiber, which can be expressed as a Stokes number.
The Stokes number is only marginally above unity and therefore, these fibers
show limited inertial effects. 
It was often hypothesized that a systematic alignment is not possible in
highly turbulent systems.
Surprisingly, we found a preferential alignment of $-0.38\pi \pm 0.05\pi$ with
respect to the inner cylinder wall.
The least probable orientation has a probability which is approximately
\SI{40}{\percent} lower as compared to the most probable one.
This alignment is persistent for all Reynolds numbers and volume fractions
tested in this chapter.
Measurements at various heights validated that this effect is not due to the
secondary flows generally present in Taylor--Couette flow.
We model the fiber orientation statistics using Jeffery's equations, which
provides a fair estimate of the shape of the alignment PDFs.
The high-speed imaging also gave access to the rotation rates of the fibers,
which was on average in the opposite direction as the IC.  
The fibers showed extreme intermittent rotation rates that are an order of
magnitude larger than the inner cylinder rotation.
In a number of ways, these finite-sized fibers behave remarkable similar to
tiny particles in turbulence.\\
\indent In \refc{chap:emulsions} we combined two immiscible fluids, in our
case water and silicone oil, in
our highly turbulent apparatus using volume fractions between
0--\SI{100}{\percent}.
Due to the tremendous amount of shear, especially close to the boundaries,
these immiscible fluids transform into a meta-stable emulsion.
We did not use an emulsifier or surfactant and therefore, the energy-input
from the turbulence provides the energy to continuously break up droplets.
Removing this energy-input by turning off the system, will make this mixture
separate in an instant.
By exploiting the known scaling of the ultimate regime of Taylor--Couette flow,
we calculated an effective viscosity, $\nue$ for the emulsion, which is
generally not possible in conventional rheometers as the mixture would
separate in an instant.
Starting from pure water, $\nue$ increases with increasing oil volume fraction
until the critical volume fraction for phase inversion has reached.
Here, the rheological properties of the mixture change in an instant as we
observe \emph{catastrophic} phase inversion.
Before the inversion, we have oil droplets in water and $\nue$ is about three
times the viscosity of water.
The morphology changes to water droplets in oil after the inversion, where
$\nue$ drops to half the viscosity of water (lower viscosity than each of the
two phases), resulting in drag reduction.
When performing dynamic measurements and therefore, change the oil volume
fraction quasi-statically during the experiment, we observe an ambivalence region.
In this ambivalence region, both morphologies are possible, depending on  the
history and the
amount of shear applied.
Using an in-situ microscopy setup, we were able to measure the droplet size in
the ambivalence region for each morphology, which revealed that water droplets
in oil are $14\times$ larger in equivalent diameter than oil droplets in water.
Larger droplets can deform easier and therefore, we expect a similar drag
reduction mechanism as with bubbly drag reduction.\\
\indent In \refc{chap:mixedbc} we use direct numerical simulations to
investigate the effect of inhomogeneous boundary conditions on
Rayleigh-B\'enard convection.
We first applied a striped pattern of insulating and conducting areas on the 
top boundary.
The area of the insulating and conducting areas are kept constant.
Only the arrangement of the areas is varied by alternating conducting and
insulating areas in stripes using various spatial frequencies.
In the extreme case, where we only have one conducting and one insulating
stripe, the top plate was practically half as effective, reducing the total
heat transport to two third of the fully conducting case.
Now changing the arrangement by increasing the amount of stripe pairs, the
transport of heat also increase.
At the largest stripe frequency, the heat transfer is very similar to the
fully conducting case, even if half of the top plate is insulating.
Extending the pattern two both sides results in similar results, however with
the lowest frequency, the system transfers heat only half as effective as a
fully conducting case.
Using a two-dimensional Fourier analysis in the horizontal plane, we are able
to see the imprint of the boundary condition pattern in the flow close to the
boundary wall.
The strength of this imprint does however decrease when we repeat this
analysis further away from the wall.
Outside of the thermal boundary layer, the imprint is indistinguishable and
the bulk flow is only exposed to an effective boundary.
Remarkably, even for the most extreme case with one large conducting and
insulating region, the imprint is not visible outside the thermal boundary
layer.
The results from this chapter demonstrate that small and even larger
imperfections in the temperature boundary conditions are barely felt in the
system dynamics.\\
\indent In \refc{chap:spanwise}, we create an inhomogeneous boundary at the
inner cylinder wall of the Taylor--Couette apparatus using p36
industrial-grade ceramic sandpaper.
In the same spirit as \refc{chap:mixedbc}, we created bands of rough and
smooth areas in the axial direction, and defined a roughness width $\tilde s$. 
When varying the size of the roughness bands, we found that we can control the
positions of the secondary flows, \textit{i.e.} the vortical structures, and
force to have \emph{more} or \emph{less} rolls in the flow.
Reducing the size of $\tilde s$ below a critical value, shows that the individual
patches do not induce single secondary flow, but create a collective outflow,
resulting in less rolls.
While the rough areas enhance the mixing and outflow, if the pattern becomes
to small (small $\tilde s$), the bulk flow only feels a boundary with an
effective average roughness.
From the torque measurements, we have revealed that there is an optimal value
for $\tilde s$ such that the angular momentum transport is enhanced.
These findings are confirmed not only for experiments, but also for
simulations at lower Taylor numbers.
This is, to our knowledge, the first time that the secondary flows could be
actively controlled in a Taylor--Couette flow.

